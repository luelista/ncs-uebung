\documentclass[a4paper,
			llpt,
			solution,
			accentcolor=tud2d,
			colorbacktitle
			]
			{tudexercise}

\usepackage[utf8]{inputenc}
%\usepackage[ngerman]{babel}
\usepackage{paralist}
\usepackage{amsmath}
\usepackage{pgfplots}
\pgfplotsset{compat=newest}
\usepgfplotslibrary{units}
%\usepgfplotslibrary{units}
\usepackage{xcolor}
%\definecolor{tud2d}{RGB}{0,78,115}
\definecolor{litegray}{gray}{0.5}

\usepackage{multicol} \setlength{\multicolsep}{0pt}

\title{Lösungsvorschlag zur vierten Hausübung}
\subtitle{Einführung in Net Centric Systems und \LaTeX, Sommersemester 2015}
\subsubtitle{Max Weller, Julian Haas, Stefan Pilot}

%\newcommand{\MiBs}{\frac{\mathrm{MiB}}{\mathrm{s}}}
\newcommand{\MiBs}{\mathrm{MiB}/\mathrm{s}}
\usepackage{multirow}
\newcommand{\8}{$\infty$}
\newcommand{\ezA}{\begin{tabular}{|c|c|c|c|c|c|}
\hline\multicolumn{2}{|c|}{\multirow{2}{*}{$\mathrm{D}^\mathrm{A}$}} & \multicolumn{4}{|c|}{Cost via}\\ \cline{3-6}\multicolumn{2}{|c|}{}& B & C & D & E\\ \hline\multirow{4}{*}{\rotatebox{90}{Destination}}}

\newcommand{\ezB}{\begin{tabular}{|c|c|c|c|c|c|}
\hline\multicolumn{2}{|c|}{\multirow{2}{*}{$\mathrm{D}^\mathrm{B}$}} & \multicolumn{4}{|c|}{Cost via}\\ \cline{3-6}\multicolumn{2}{|c|}{}& A & C & D & E\\ \hline\multirow{4}{*}{\rotatebox{90}{Destination}}}

\newcommand{\ezC}{\begin{tabular}{|c|c|c|c|c|c|}
\hline\multicolumn{2}{|c|}{\multirow{2}{*}{$\mathrm{D}^\mathrm{C}$}} & \multicolumn{4}{|c|}{Cost via}\\ \cline{3-6}\multicolumn{2}{|c|}{}& A & B & D & E\\ \hline\multirow{4}{*}{\rotatebox{90}{Destination}}}

\newcommand{\ezD}{\begin{tabular}{|c|c|c|c|c|c|}
\hline\multicolumn{2}{|c|}{\multirow{2}{*}{$\mathrm{D}^\mathrm{D}$}} & \multicolumn{4}{|c|}{Cost via}\\ \cline{3-6}\multicolumn{2}{|c|}{}& A & B & C & E\\ \hline\multirow{4}{*}{\rotatebox{90}{Destination}}}

\newcommand{\ezE}{\begin{tabular}{|c|c|c|c|c|c|}
\hline\multicolumn{2}{|c|}{\multirow{2}{*}{$\mathrm{D}^\mathrm{E}$}} & \multicolumn{4}{|c|}{Cost via}\\ \cline{3-6}\multicolumn{2}{|c|}{}& A & B & C & D\\ \hline\multirow{4}{*}{\rotatebox{90}{Destination}}}

\newcommand{\ze}{\end{tabular}}

\newcommand{\upd}{\begin{tabular}{c|c}}
\begin{document}
\maketitle
\section{} \section{} \section{} \section{} \section{}
\section{Multiple Choice}
\subsection{Which is the lowest layer that uses traffic regulation mechanisms to keep a fast transmitter from drowning a slow receiver in data?}
\begin{compactenum}
\item[d)] Data Link Layer
\end{compactenum}
\subsection{Which layer is responsible for determining how packets are routed from source to destination?}
\begin{compactenum}
\item[b)] Network Layer
\end{compactenum}
\subsection{Which kind of routing algorithms are variable according to the current network status?}
Das kommt darauf an, wie man "Status" definiert. Dynamische Routingalgorithmen sind in der Lage, den Ausfall von Knoten oder Verbindungen zu kompensieren. Adaptive Routingalgorithmen berücksichtigen auch die Auslastung einzelner Knoten und Verbindungen. Der Übergang ist allerdings fließend und hängt auch von der Metrik ab.
\begin{compactenum}
\item[a)] Adaptive routing algorithms
\item[d)] Dynamic routing algorithms
\end{compactenum}
\subsection{For which algorithms the count-to-infinity problem does not occur?}
\begin{compactenum}
\item[a)] LSR
\item[c)] BGP
\item[d)] OSPF
\item[f)] DSR
\end{compactenum}
\subsection{When a connection is established, a route from the source machine to the destination machine is chosen as a part of the connection setup and stored in tables inside the routers. For which connection is the statement valid?}
\begin{compactenum}
\item[b)]{Connection-oriented Services}
\end{compactenum}
\subsection{Which of the algorithms throws away tokens when a bucket fills up but never discards packets?}
\begin{compactenum}
\item Token Bucket
\end{compactenum}
\subsection{Which of the fields belong only to IPv6?}
\begin{compactenum}
\item[c)] Next header
\item[d)] Payload length
\item[f)] Hop limit
\end{compactenum}
\subsection{How do you call a connection that allows traffic either way but only one way at a time?}
\begin{compactenum}
\item[c)]{half duplex}
\end{compactenum}
\subsection{According to DVR all update messages are sent...}
\begin{compactenum}
\item[c)]{only to neighbouring nodes}
\end{compactenum}
\end{document}
