\documentclass[a4paper,
			llpt,
			solution,
			accentcolor=tud2d,
			colorbacktitle
			]
			{tudexercise}

\usepackage[utf8]{inputenc}
\usepackage[ngerman]{babel}
\usepackage{paralist}
\usepackage{amsmath}
\usepackage{pgfplots}
\pgfplotsset{compat=newest}
\usepgfplotslibrary{units}
%\usepgfplotslibrary{units}

\usepackage{multicol} \setlength{\multicolsep}{0pt}

\title{Lösungsvorschlag zur dritten Hausübung}
\subtitle{Einführung in Net Centric Systems und \LaTeX, Sommersemester 2015}
\subsubtitle{Max Weller, Julian Haas, Stefan Pilot}

%\newcommand{\MiBs}{\frac{\mathrm{MiB}}{\mathrm{s}}}
\newcommand{\MiBs}{\mathrm{MiB}/\mathrm{s}}
\usepackage{multirow}
\newcommand{\8}{$\infty$}
\newcommand{\ezA}{\begin{tabular}{|c|c|c|c|c|c|}
\hline\multicolumn{2}{|c|}{\multirow{2}{*}{$\mathrm{D}^\mathrm{A}$}} & \multicolumn{4}{|c|}{Cost via}\\ \cline{3-6}\multicolumn{2}{|c|}{}& B & C & D & E\\ \hline\multirow{4}{*}{\rotatebox{90}{Destination}}}

\newcommand{\ezB}{\begin{tabular}{|c|c|c|c|c|c|}
\hline\multicolumn{2}{|c|}{\multirow{2}{*}{$\mathrm{D}^\mathrm{B}$}} & \multicolumn{4}{|c|}{Cost via}\\ \cline{3-6}\multicolumn{2}{|c|}{}& A & C & D & E\\ \hline\multirow{4}{*}{\rotatebox{90}{Destination}}}

\newcommand{\ezC}{\begin{tabular}{|c|c|c|c|c|c|}
\hline\multicolumn{2}{|c|}{\multirow{2}{*}{$\mathrm{D}^\mathrm{C}$}} & \multicolumn{4}{|c|}{Cost via}\\ \cline{3-6}\multicolumn{2}{|c|}{}& A & B & D & E\\ \hline\multirow{4}{*}{\rotatebox{90}{Destination}}}

\newcommand{\ezD}{\begin{tabular}{|c|c|c|c|c|c|}
\hline\multicolumn{2}{|c|}{\multirow{2}{*}{$\mathrm{D}^\mathrm{D}$}} & \multicolumn{4}{|c|}{Cost via}\\ \cline{3-6}\multicolumn{2}{|c|}{}& A & B & C & E\\ \hline\multirow{4}{*}{\rotatebox{90}{Destination}}}

\newcommand{\ezE}{\begin{tabular}{|c|c|c|c|c|c|}
\hline\multicolumn{2}{|c|}{\multirow{2}{*}{$\mathrm{D}^\mathrm{E}$}} & \multicolumn{4}{|c|}{Cost via}\\ \cline{3-6}\multicolumn{2}{|c|}{}& A & B & C & D\\ \hline\multirow{4}{*}{\rotatebox{90}{Destination}}}

\newcommand{\ze}{\end{tabular}}

\newcommand{\upd}{\begin{tabular}{c|c}}
\begin{document}

\maketitle
\section{ -- 3.1}
\subsection{3.1.1}
%\ezA
%&B&&&&\\ \cline{2-6}
%&C&&&&\\ \cline{2-6}
%&D&&&&\\ \cline{2-6}
%&E&&&&\\ \hline
%\end{tabular}
%\\
%%% 3.1.1. %%%
\begin{multicols}{3}
Der Graph:\\
\begin{center}
\begin{tikzpicture}[-,
					%>=stealth',
					%shorten >=1pt,
					auto,
					node distance=2cm,
					thick,
					main node/.style={circle,draw}]

  \node[main node] (A) {A};
  \node[main node] (B) [left of=A] {B};
  \node[main node] (C) [right of=A] {C};
  \node[main node] (D) [below of=C] {D};
  \node[main node] (E) [below of=B] {E};
  
  \path[every node/.style={}]
    (A) edge node {7} (B)
        edge node {5} (C)
        edge node {1} (D)
        edge node {2} (E)
    (B) edge node {4} (E)
  	(C) edge node {1} (D)
  	(D) edge node {3} (E);
\end{tikzpicture}
\end{center}
\columnbreak
Routingtabellen nach der Initialisierungsphase (dem nicht in der Liste auftauchenden Knoten gehört die Tabelle): \\
%\begin{multicols}{3}
\begin{tabular}{c|c|c}
Destination & Cost & Via \\ \hline
B           & 7    & B   \\
C           & 5    & C   \\
D           & 1    & D   \\
E           & 2    & E   \\
\end{tabular}
\begin{tabular}{c|c|c}
Destination & Cost & Via \\ \hline
A           & 7    & A   \\
C           &      &     \\
D           &      &     \\
E           & 4    & E   \\
\end{tabular}
\begin{tabular}{c|c|c}
Destination & Cost & Via \\ \hline
A           & 5    & A   \\
B           &      &     \\
D           & 1    & D   \\
E           &      &     \\
\end{tabular}
\begin{tabular}{c|c|c}
Destination & Cost & Via \\ \hline
A           & 1    & A   \\
B           &      &     \\
C           & 1    & C   \\
E           & 3    & E   \\
\end{tabular}
\begin{tabular}{c|c|c}
Destination & Cost & Via \\ \hline
A           & 2    & A   \\
B           & 4    & B   \\
C           &      &     \\
D           & 3    & D   \\
\end{tabular}
\end{multicols}
Die ersten update messages:
\begin{center}
\begin{tabular}{|c|c|@{}c@{}|@{}c@{}|@{}c@{}|@{}c@{}|@{}c@{}|}
\hline \multicolumn{2}{|c|}{Update} & \multicolumn{5}{|c|}{sent to:} \\ \cline{3-7} \multicolumn{2}{|c|}{Messages} & A & B & C & D & E \\ \hline \multirow{5}{*}{\rotatebox{90}{sent by:}}
& A & &
\upd
to C & 5 \\ to D & 1 \\ to E & 2 \\
\end{tabular} &
\upd
to B & 7 \\ to D & 1 \\ to E & 2 \\
\end{tabular} &
\upd
to B & 7 \\ to C & 5 \\ to E & 2 \\
\end{tabular} &
\upd
to B & 7 \\ to C & 5 \\ to D & 1 \\
\end{tabular} \\ \cline{2-7}
& B &
\upd
to E & 4 \\
\end{tabular} & & & &
\upd
to A & 7 \\
\end{tabular} \\ \cline{2-7}
& C &
\upd
to D & 1
\end{tabular} & & &
\upd
to A & 5
\end{tabular} & \\ \cline{2-7}
& D &
\upd
to C & 1 \\ to E & 3 \\
\end{tabular} & &
\upd
to A & 1 \\ to E & 3 \\
\end{tabular} & &
\upd
to A & 1 \\ to C & 1 \\
\end{tabular} \\ \cline{2-7}
& E &
\upd
to B & 4 \\ to D & 3 \\
\end{tabular} &
\upd
to A & 2 \\ to D & 3 \\
\end{tabular} & &
\upd
to A & 2 \\ to B & 4 \\
\end{tabular} &
\\ \hline
\end{tabular}
\end{center}
\clearpage
%%% 3.1.2 %%%
\subsection{3.1.2}
Routingtabellen der ersten Iteration:\\
\begin{multicols}{3}
\begin{tabular}{c|c|c}
Destination & Cost & Via \\ \hline
B           & 6    & E   \\
C           & 2    & D   \\
D           & 1    & D   \\
E           & 2    & E   \\
\end{tabular}
\vfill
\columnbreak
\begin{tabular}{c|c|c}
Destination & Cost & Via \\ \hline
A           & 6    & A   \\
C           & 11   & E   \\
D           & 7    & E   \\
E           & 4    & E   \\
\end{tabular}
\vfill
\columnbreak
\begin{tabular}{c|c|c}
Destination & Cost & Via \\ \hline
A           & 2    & D   \\
B           & 9    & D   \\
D           & 1    & D   \\
E           & 4    & D   \\
\end{tabular}
\end{multicols}
~\\[0.1ex]
\begin{multicols}{3}
\begin{tabular}{c|c|c}
Destination & Cost & Via \\ \hline
A           & 1    & A   \\
B           & 7    & E   \\
C           & 1    & C   \\
E           & 3    & E   \\
\end{tabular}
\vfill
\columnbreak
\begin{tabular}{c|c|c}
Destination & Cost & Via \\ \hline
A           & 2    & A   \\
B           & 4    & B   \\
C           & 4    & D   \\
D           & 3    & D   \\
\end{tabular}
\end{multicols}
~\\
Update Messages Iteration 1:
\begin{center}
\begin{tabular}{|c|c|@{}c@{}|@{}c@{}|@{}c@{}|@{}c@{}|@{}c@{}|}
\hline \multicolumn{2}{|c|}{Update} & \multicolumn{5}{|c|}{sent to:} \\ \cline{3-7} \multicolumn{2}{|c|}{Messages} & A & B & C & D & E \\ \hline \multirow{5}{*}{\rotatebox{90}{sent by:}}
&A&&
\upd      &   \\ to C & 2\ze &
\upd to B & 6 \\      &  \ze &
\upd to B & 6 \\ to C & 2\ze &
\upd to B & 6 \\ to C & 2\ze \\ \cline{2-7}
&B&
\upd to C & 11 \\ to D & 7 \ze &&&&
\upd to C & 11 \\ to D & 7 \ze \\ \cline{2-7}
&C&
\upd      &   \\ to B & 9 \\ to E & 4\ze &&&
\upd to A & 2 \\ to B & 9 \\ to E & 4\ze & \\ \cline{2-7}
&D&
\upd to B & 7\ze &&
\upd to B & 7\ze &&
\upd to B & 7\ze \\ \cline{2-7}
&E&
\upd to C & 4\ze &
\upd to C & 4\ze &
\upd \ze &
\upd to C & 4\ze &\\ \hline
\end{tabular}
\end{center}
Routingtabellen der zweiten Iteration:
\begin{multicols}{3}
\begin{tabular}{c|c|c}
Destination & Cost & Via \\ \hline
B           & 6    & E   \\
C           & 2    & D   \\
D           & 1    & D   \\
E           & 2    & E   \\
\end{tabular}
\vfill \columnbreak
\begin{tabular}{c|c|c}
Destination & Cost & Via \\ \hline
A           & 6    & A   \\
C           & 8    & A   \\
D           & 7    & E   \\
E           & 4    & E   \\
\end{tabular}
\vfill \columnbreak
\begin{tabular}{c|c|c}
Destination & Cost & Via \\ \hline
A           & 2    & D   \\
B           & 8    & D   \\
D           & 1    & D   \\
E           & 4    & D   \\
\end{tabular}
\end{multicols}
~\\
\begin{multicols}{3}
\begin{tabular}{c|c|c}
Destination & Cost & Via \\ \hline
A           & 1    & A   \\
B           & 7    & E   \\
C           & 1    & C   \\
E           & 3    & E   \\
\end{tabular}
\vfill \columnbreak
\begin{tabular}{c|c|c}
Destination & Cost & Via \\ \hline
A           & 2    & A   \\
B           & 4    & B   \\
C           & 4    & D   \\
D           & 3    & D   \\
\end{tabular}
\end{multicols}
~\\
update messages der zweiten Iteration:
\begin{center}
\begin{tabular}{|c|c|@{}c@{}|@{}c@{}|@{}c@{}|@{}c@{}|@{}c@{}|}
\hline \multicolumn{2}{|c|}{Update} & \multicolumn{5}{|c|}{sent to:} \\ \cline{3-7} \multicolumn{2}{|c|}{Messages} & A & B & C & D & E \\ \hline \multirow{5}{*}{\rotatebox{90}{sent by:}}
&A&&&&&\\ \cline{2-7}
&B&\upd to C & 8\ze &&&\upd to C & 8\ze &\\ \cline{2-7}
&C&\upd to B & 8\ze &&&\upd to B & 8\ze &\\ \cline{2-7}
&D&&&&&\\ \cline{2-7}
&E&&&&&\\ \hline
\end{tabular}
\end{center}
Nach der zweiten Iteration werden kennen alle Knoten die optimalen Routen.
\clearpage
\subsection{3.1.3}
\begin{multicols}{3}
Der Graph:\\
\begin{tikzpicture}[-,
					%>=stealth',
					%shorten >=1pt,
					auto,
					node distance=2cm,
					thick,
					main node/.style={circle,draw}]

  \node[main node] (A) {A};
  \node[main node] (B) [left of=A] {B};
  \node[main node] (C) [right of=A] {C};
  \node[main node] (D) [below of=C] {D};
  \node[main node] (E) [below of=B] {E};
  
  \path[every node/.style={}]
    (A) edge node {7} (B)
        edge node {\textcolor{red}{1}} (C)
        edge node {1} (D)
        edge node {2} (E)
    (B) edge node {4} (E)
  	(C) edge node {1} (D)
  	(D) edge node {3} (E);
\end{tikzpicture}
\vfill
\columnbreak
Distanztabellen nach der Initialisierungsphase:\\\\
\begin{tabular}{c|c|c}
Destination & Cost & Via \\ \hline
B           & 7    & B   \\
C           & \textcolor{red}{1}    & C   \\
D           & 1    & D   \\
E           & 2    & E   \\
\end{tabular}
\begin{tabular}{c|c|c}
Destination & Cost & Via \\ \hline
A           & 7    & A   \\
C           &      &     \\
D           &      &     \\
E           & 4    & E   \\
\end{tabular}
\begin{tabular}{c|c|c}
Destination & Cost & Via \\ \hline
A           & \textcolor{red}{1}    & A   \\
B           &      &     \\
D           & 1    & D   \\
E           &      &     \\
\end{tabular}
\begin{tabular}{c|c|c}
Destination & Cost & Via \\ \hline
A           & 1    & A   \\
B           &      &     \\
C           & 1    & C   \\
E           & 3    & E   \\
\end{tabular}
\begin{tabular}{c|c|c}
Destination & Cost & Via \\ \hline
A           & 2    & A   \\
B           & 4    & B   \\
C           &      &     \\
D           & 3    & D   \\
\end{tabular}
\end{multicols}
Die ersten update messages:
\begin{center}
\begin{tabular}{|c|c|@{}c@{}|@{}c@{}|@{}c@{}|@{}c@{}|@{}c@{}|}
\hline \multicolumn{2}{|c|}{Update} & \multicolumn{5}{|c|}{sent to:} \\ \cline{3-7} \multicolumn{2}{|c|}{Messages} & A & B & C & D & E \\ \hline \multirow{5}{*}{\rotatebox{90}{sent by:}}
& A & &
\upd
to C & \textcolor{red}{1} \\ to D & 1 \\ to E & 2 \\
\end{tabular} &
\upd
to B & 7 \\ to D & 1 \\ to E & 2 \\
\end{tabular} &
\upd
to B & 7 \\ to C & \textcolor{red}{1} \\ to E & 2 \\
\end{tabular} &
\upd
to B & 7 \\ to C & \textcolor{red}{1} \\ to D & 1 \\
\end{tabular} \\ \cline{2-7}
& B &
\upd
to E & 4 \\
\end{tabular} & & & &
\upd
to A & 7 \\
\end{tabular} \\ \cline{2-7}
& C &
\upd
to D & 1
\end{tabular} & & &
\upd
to A & \textcolor{red}{1}
\end{tabular} & \\ \cline{2-7}
& D &
\upd
to C & 1 \\ to E & 3 \\
\end{tabular} & &
\upd
to A & 1 \\ to E & 3 \\
\end{tabular} & &
\upd
to A & 1 \\ to C & 1 \\
\end{tabular} \\ \cline{2-7}
& E &
\upd
to B & 4 \\ to D & 3 \\
\end{tabular} &
\upd
to A & 2 \\ to D & 3 \\
\end{tabular} & &
\upd
to A & 2 \\ to B & 4 \\
\end{tabular} &
\\ \hline
\end{tabular}
\end{center}
Routingtabellen der ersten Iteration:
\begin{multicols}{3}
\begin{tabular}{c|c|c}
Destination & Cost & Via \\ \hline
B           & 6    & E   \\
C           & \textcolor{red}{1}    & \textcolor{red}{C}   \\
D           & 1    & D   \\
E           & 2    & E   \\
\end{tabular}
\begin{tabular}{c|c|c}
Destination & Cost & Via \\ \hline
A           & 6    & A   \\
C           & \textcolor{red}{8}   & \textcolor{red}{E}   \\
D           & 7    & E   \\
E           & 4    & E   \\
\end{tabular}
\begin{tabular}{c|c|c}
Destination & Cost & Via \\ \hline
A           & \textcolor{red}{1}    & \textcolor{red}{A}   \\
B           & \textcolor{red}{8}    & \textcolor{red}{A}   \\
D           & 1    & D   \\
E           & \textcolor{red}{3}    & \textcolor{red}{A}   \\
\end{tabular}
\begin{tabular}{c|c|c}
Destination & Cost & Via \\ \hline
A           & 1    & A   \\
B           & 7    & E   \\
C           & 1    & C   \\
E           & 3    & E   \\
\end{tabular}
\begin{tabular}{c|c|c}
Destination & Cost & Via \\ \hline
A           & 2    & A   \\
B           & 4    & B   \\
C           & \textcolor{red}{3}    & \textcolor{red}{A}   \\
D           & 3    & D   \\
\end{tabular}
\end{multicols}
update messages der ersten Iteration:
\begin{center}
\begin{tabular}{|c|c|@{}c@{}|@{}c@{}|@{}c@{}|@{}c@{}|@{}c@{}|}
\hline \multicolumn{2}{|c|}{Update} & \multicolumn{5}{|c|}{sent to:} \\ \cline{3-7} \multicolumn{2}{|c|}{Messages} & A & B & C & D & E \\ \hline \multirow{5}{*}{\rotatebox{90}{sent by:}}
&A&&
\upd      &   \\ \colorbox{red}{\textcolor{red}{to C}} & \colorbox{red}{\textcolor{red}{2}}\ze &
\upd to B & 6 \\      &  \ze &
\upd to B & 6 \\ \colorbox{red}{\textcolor{red}{to C}} & \colorbox{red}{\textcolor{red}{2}}\ze &
\upd to B & 6 \\ \colorbox{red}{\textcolor{red}{to C}} & \colorbox{red}{\textcolor{red}{2}}\ze \\ \cline{2-7}
&B&
\upd to C & \textcolor{red}{8} \\ to D & 7 \ze &&&&
\upd to C  & \textcolor{red}{8} \\ to D & 7 \ze \\ \cline{2-7}
&C&
\upd      &   \\ to B & \textcolor{red}{8} \\ to E & \textcolor{red}{3}\ze &&&
\upd \colorbox{red}{\textcolor{red}{to A}} & \colorbox{red}{\textcolor{red}{2}} \\ to B & \textcolor{red}{8} \\ to E & \textcolor{red}{3}\ze & \\ \cline{2-7}
&D&
\upd to B & 7\ze &&
\upd to B & 7\ze &&
\upd to B & 7\ze \\ \cline{2-7}
&E&
\upd to C & \textcolor{red}{3}\ze &
\upd to C & \textcolor{red}{3}\ze &
\upd \ze &
\upd to C & \textcolor{red}{3}\ze &\\ \hline
\end{tabular}
\end{center}
Nach der ersten Iteration werden keine update messages mehr versandt.

\subsection{3.1.4}
Wahrscheinlich will man hier hören, dass ein Count-To-Infinity eintritt - ich verstehe aber nicht, warum das so ist.
\colorbox{red}{\textcolor{white}{SOLUTION MISSING!}}
\subsection{3.1.5}
\colorbox{red}{\textcolor{white}{SOLUTION MISSING!}}
\subsection{3.1.6}
No it isn't. \colorbox{red}{\textcolor{white}{EXPLANATION HERE!}}
\clearpage
\section{ -- 3.2}
\subsection{3.2.1}
Das Link-State-Routing basiert darauf, dass jeder Knoten die komplette Netzwerktopologie kennt. Alle Knoten verschicken ihre Adjazenzliste an ihre Nachbarn; jeder Knoten verteilt alle Adjazenzlisten, die er erhält, weiter an seine Nachbarn. So kann jeder Knoten einen Graphen erzeugen, der das Netzwerk abbildet - sobald er von zwei Knoten die Information erhält, dass sie mit dem jeweils anderen verbunden sind, fügt er dem Graphen eine Verbindung hinzu. Anhand dieses Graphen kann jeder Knoten per Dijkstra-Algorithmus den kürzesten Pfad zu allen anderen Knoten bestimmen und danach routen. Das edge weight ist dabei abhängig von der Implementierung und kann zum Beispiel Bandbreiteninformationen erhalten. 

\subsection{3.2.2}
\begin{enumerate}
\item
Wir gehen davon aus, dass ein Knoten eine Verbindung erst anerkennt, wenn er von beiden Knoten erfahren hat, dass sie mit dem jeweils anderen verbunden sind.
\\
\begin{multicols}{2}
Wissensstand der Knoten, nachdem sie ihre Nachbarknoten gefunden haben:
\\
Wissensstand der Knoten nach der ersten Nachrichtenwelle:
\end{multicols}
\begin{multicols}{2}
%\begin{table}
%\caption{Den Knoten bekannte Netzwerkteile, Iteration 0}

\begin{tabular}{|c|c|}
\hline
Knoten & Dem Knoten bekannter Teil des Netzwerks\\ \hline
A & \begin{tikzpicture}[-,
					%>=stealth',
					%shorten >=1pt,
					auto,
					node distance=1.2cm,
					thick,
					main node/.style={circle,draw}]

  \node[main node] (A) {A};
  \node[main node] (B) [left of=A] {B};
  \node[main node] (E) [right of=A] {E};
  
  \path[every node/.style={}]
    (A) edge node {10} (E)
        edge node {4} (B);
\end{tikzpicture}
\\ \hline

B & \begin{tikzpicture}[-,
					%>=stealth',
					%shorten >=1pt,
					auto,
					node distance=1.2cm,
					thick,
					main node/.style={circle,draw}]

  \node[main node] (B) {B};
  \node[main node] (A) [left of=B] {A};
  \node[main node] (F) [below of=B] {F};
  \node[main node] (D) [right of=B] {D};
  
  \path[every node/.style={}]
    (B) edge node {4} (A)
        edge node {2} (F)
        edge node {5} (D);
\end{tikzpicture}\\ \hline
C & \begin{tikzpicture}[-,
					%>=stealth',
					%shorten >=1pt,
					auto,
					node distance=1.2cm,
					thick,
					main node/.style={circle,draw}]

  \node[main node] (A) {C};
  \node[main node] (D) [left of=A] {D};
  \node[main node] (G) [right of=A] {G};
  
  \path[every node/.style={}]
    (A) edge node {5} (D)
        edge node {7} (G);
\end{tikzpicture}\\ \hline
D & \begin{tikzpicture}[-,
					%>=stealth',
					%shorten >=1pt,
					auto,
					node distance=1.2cm,
					thick,
					main node/.style={circle,draw}]

  \node[main node] (D) {D};
  \node[main node] (B) [below left of=D] {B};
  \node[main node] (G) [above right of=D] {G};
  \node[main node] (F) [below right of=D] {F};
  \node[main node] (C) [above left of=D] {C};
  
  \path[every node/.style={}]
    (D) edge node {5} (B)
        edge node {4} (C)
        edge node {6} (G)
        edge node {3} (F);
\end{tikzpicture}\\ \hline
E & \begin{tikzpicture}[-,
					%>=stealth',
					%shorten >=1pt,
					auto,
					node distance=1.2cm,
					thick,
					main node/.style={circle,draw}]

  \node[main node] (E) {E};
  \node[main node] (A) [left of=E] {A};
  \node[main node] (F) [right of=E] {F};
  
  \path[every node/.style={}]
    (E) edge node {10} (A)
        edge node {4} (F);
\end{tikzpicture}\\ \hline
F &  \begin{tikzpicture}[-,
					%>=stealth',
					%shorten >=1pt,
					auto,
					node distance=1.2cm,
					thick,
					main node/.style={circle,draw}]

  \node[main node] (F) {F};
  \node[main node] (E) [left of=F] {E};
  \node[main node] (B) [below of=F] {B};
  \node[main node] (D) [right of=F] {D};
  
  \path[every node/.style={}]
    (F) edge node {4} (E)
        edge node {2} (B)
        edge node {3} (D);
\end{tikzpicture}\\ \hline
G &\begin{tikzpicture}[-,
					%>=stealth',
					%shorten >=1pt,
					auto,
					node distance=1.2cm,
					thick,
					main node/.style={circle,draw}]

  \node[main node] (G) {G};
  \node[main node] (D) [left of=G] {D};
  \node[main node] (C) [right of=G] {C};
  
  \path[every node/.style={}]
    (G) edge node {7} (C)
        edge node {6} (D);
\end{tikzpicture} \\ \hline
\end{tabular}

%\end{table}

%\begin{table}
%\caption{Den Knoten bekannte Netzwerkteile, Iteration 1}

\begin{tabular}{|c|c|}
\hline
Knoten & Dem Knoten bekannter Teil des Netzwerks\\ \hline
A & \begin{tikzpicture}[-,
					%>=stealth',
					%shorten >=1pt,
					auto,
					node distance=1.2cm,
					thick,
					main node/.style={circle,draw}]

  \node[main node] (A) {A};
  \node[main node] (B) [left of=A] {B};
  \node[main node] (E) [right of=A] {E};
  
  \path[every node/.style={}]
    (A) edge node {10} (E)
        edge node {4} (B);
\end{tikzpicture}
\\ \hline

B & \begin{tikzpicture}[-,
					%>=stealth',
					%shorten >=1pt,
					auto,
					node distance=1.5cm,
					thick,
					main node/.style={circle,draw}]

  \node[main node] (B) {B};
  \node[main node] (A) [left of=B] {A};
  \node[main node] (D) [right of=B] {D};
  \node[main node] (F) [below of=B] {F};

  
  \path[every node/.style={}]
    (B) edge node {4} (A)
        edge node {2} (F)
        edge node {5} (D)
    (D) edge node {3} (F);
\end{tikzpicture}\\ \hline
C & \begin{tikzpicture}[-,
					%>=stealth',
					%shorten >=1pt,
					auto,
					node distance=1.5cm,
					thick,
					main node/.style={circle,draw}]

  \node[main node] (D) {D};
  \node[main node] (C) [above right of=D] {C};
  \node[main node] (G) [below right of=D] {G};
  
  \path[every node/.style={}]
    (D) edge node {5} (C)
        edge node {6} (G)
    (C) edge node {7} (G);
\end{tikzpicture}\\ \hline
D & \begin{tikzpicture}[-,
					%>=stealth',
					%shorten >=1pt,
					auto,
					node distance=1.5cm,
					thick,
					main node/.style={circle,draw}]

  \node[main node] (D) {D};
  \node[main node] (B) [above left of=D] {B};
  \node[main node] (G) [below right of=D] {G};
  \node[main node] (F) [below left of=D] {F};
  \node[main node] (C) [above right of=D] {C};
  
  \path[every node/.style={}]
    (D) edge node {5} (B)
        edge node {4}  (C)
        edge node {6}  (G)
        edge node {3}  (F)
    (F) edge node {2}  (B)
    (C) edge node {7}  (G);
\end{tikzpicture}\\ \hline
E & \begin{tikzpicture}[-,
					%>=stealth',
					%shorten >=1pt,
					auto,
					node distance=1.2cm,
					thick,
					main node/.style={circle,draw}]

  \node[main node] (E) {E};
  \node[main node] (A) [left of=E] {A};
  \node[main node] (F) [right of=E] {F};
  
  \path[every node/.style={}]
    (E) edge node {10} (A)
        edge node {4} (F);
\end{tikzpicture}\\ \hline
F &  \begin{tikzpicture}[-,
					%>=stealth',
					%shorten >=1pt,
					auto,
					node distance=1.2cm,
					thick,
					main node/.style={circle,draw}]

  \node[main node] (F) {F};
  \node[main node] (E) [left of=F] {E};
  \node[main node] (B) [below of=F] {B};
  \node[main node] (D) [right of=F] {D};
  
  \path[every node/.style={}]
    (F) edge node {4} (E)
        edge node {2} (B)
        edge node {3} (D);
\end{tikzpicture}\\ \hline
G &\begin{tikzpicture}[-,
					%>=stealth',
					%shorten >=1pt,
					auto,
					node distance=1.5cm,
					thick,
					main node/.style={circle,draw}]

  \node[main node] (D) {D};
  \node[main node] (C) [above right of=D] {C};
  \node[main node] (G) [below right of=D] {G};
  
  \path[every node/.style={}]
    (D) edge node {5} (C)
        edge node {6} (G)
    (C) edge node {7} (G);
\end{tikzpicture}\\ \hline
\end{tabular}
\end{multicols}
Wissenstand der Knoten nach der zweiten Nachrichtenwelle:\\
\begin{multicols}{2}
\begin{tabular}{|c|c|}
\hline
Knoten & Dem Knoten bekannter Teil des Netzwerks\\ \hline
A & \begin{tikzpicture}[-,
					%>=stealth',
					%shorten >=1pt,
					auto,
					node distance=1.5cm,
					thick,
					main node/.style={circle,draw}]

  \node[main node] (D) {D};
  \node[main node] (B) [above left of=D] {B};
  \node[main node] (F) [below left of=D] {F};
  \node[main node] (A) [left of=B] {A};
  \node[main node] (E) [left of=F] {E};
  
  \path[every node/.style={}]
    (D) edge node {3} (F)
    (F) edge node {4} (E)
        edge node {2} (B)
    (B) edge node {4} (A)
        edge node {5} (D)
    (E) edge node {10}(A)
;
\end{tikzpicture}
\\ \hline

B &\begin{tikzpicture}[-,
					%>=stealth',
					%shorten >=1pt,
					auto,
					node distance=1.5cm,
					thick,
					main node/.style={circle,draw}]

  \node[main node] (D) {D};
  \node[main node] (B) [above left of=D] {B};
  \node[main node] (F) [below left of=D] {F};
  \node[main node] (A) [left of=B] {A};
  \node[main node] (E) [left of=F] {E};
  \node[main node] (C) [above right of=D] {C};
  \node[main node] (G) [below right of=D] {G}; 
  
  \path[every node/.style={}]
    (D) edge node {5} (B)
        edge node {3} (F)
        edge node {5} (C)
        edge node {6} (G)
    (F) edge node {4} (E)
        edge node {2} (B)
    (B) edge node {4} (A)
    (A) edge node {10}(E)
    (C) edge node {7} (G)
;
\end{tikzpicture}\\ \hline
C & \begin{tikzpicture}[-,
					%>=stealth',
					%shorten >=1pt,
					auto,
					node distance=1.8cm,
					thick,
					main node/.style={circle,draw}]

  \node[main node] (D) {D};
  \node[main node] (B) [below left of=D] {B};
  \node[main node] (G) [above right of=D] {G};
  \node[main node] (F) [below right of=D] {F};
  \node[main node] (C) [above left of=D] {C};
  
  \path[every node/.style={}]
    (D) edge node {10} (B)
        edge node {4}  (C)
        edge node {6}  (G)
        edge node {3}  (F)
    (B) edge node {2}  (F)
    (C) edge node {7}  (G);
\end{tikzpicture}\\ \hline
D &\begin{tikzpicture}[-,
					%>=stealth',
					%shorten >=1pt,
					auto,
					node distance=1.5cm,
					thick,
					main node/.style={circle,draw}]

  \node[main node] (D) {D};
  \node[main node] (B) [above left of=D] {B};
  \node[main node] (F) [below left of=D] {F};
  \node[main node] (A) [left of=B] {A};
  \node[main node] (E) [left of=F] {E};
  \node[main node] (C) [above right of=D] {C};
  \node[main node] (G) [below right of=D] {G}; 
  
  \path[every node/.style={}]
    (D) edge node {5} (B)
        edge node {3} (F)
        edge node {5} (C)
        edge node {6} (G)
    (F) edge node {4} (E)
        edge node {2} (B)
    (B) edge node {4} (A)
    (A) edge node {10}(E)
    (C) edge node {7} (G)
;
\end{tikzpicture}\\ \hline \end{tabular}
\vfill
\columnbreak
\begin{tabular}{|c|c|}
\hline
Knoten & Dem Knoten bekannter Teil des Netzwerks\\ \hline
E & \begin{tikzpicture}[-,
					%>=stealth',
					%shorten >=1pt,
					auto,
					node distance=1.5cm,
					thick,
					main node/.style={circle,draw}]

  \node[main node] (D) {D};
  \node[main node] (B) [above left of=D] {B};
  \node[main node] (F) [below left of=D] {F};
  \node[main node] (A) [left of=B] {A};
  \node[main node] (E) [left of=F] {E};
  
  \path[every node/.style={}]
    (D) edge node {3} (F)
    (F) edge node {4} (E)
        edge node {2} (B)
    (B) edge node {4} (A)
        edge node {5} (D)
    (E) edge node {10}(A)
;
\end{tikzpicture}\\ \hline
F &  \begin{tikzpicture}[-,
					%>=stealth',
					%shorten >=1pt,
					auto,
					node distance=1.5cm,
					thick,
					main node/.style={circle,draw}]

  \node[main node] (D) {D};
  \node[main node] (B) [above left of=D] {B};
  \node[main node] (F) [below left of=D] {F};
  \node[main node] (A) [left of=B] {A};
  \node[main node] (E) [left of=F] {E};
  \node[main node] (C) [above right of=D] {C};
  \node[main node] (G) [below right of=D] {G}; 
  
  \path[every node/.style={}]
    (D) edge node {5} (B)
        edge node {3} (F)
        edge node {5} (C)
        edge node {6} (G)
    (F) edge node {4} (E)
        edge node {2} (B)
    (B) edge node {4} (A)
    (A) edge node {10}(E)
    (C) edge node {7} (G)
;
\end{tikzpicture}\\ \hline
G &\begin{tikzpicture}[-,
					%>=stealth',
					%shorten >=1pt,
					auto,
					node distance=1.8cm,
					thick,
					main node/.style={circle,draw}]

  \node[main node] (D) {D};
  \node[main node] (B) [below left of=D] {B};
  \node[main node] (G) [above right of=D] {G};
  \node[main node] (F) [below right of=D] {F};
  \node[main node] (C) [above left of=D] {C};
  
  \path[every node/.style={}]
    (D) edge node {10} (B)
        edge node {4}  (C)
        edge node {6}  (G)
        edge node {3}  (F)
    (B) edge node {2}  (F)
    (C) edge node {7}  (G);
\end{tikzpicture}\\ \hline
\end{tabular}
\end{multicols}
%\end{table}
In der folgenden dritten Iteration kennt jeder Knoten das gesamte Netzwerk.
\item
Das wurde in Teilaufgabe 2.1 erläutert.
\end{enumerate}
\subsection{3.2.3}
Nein, aber ich kann nicht erklären, warum das so ist. \colorbox{red}{\textcolor{white}{SOLUTION MISSING!}}
\subsection{3.2.4}
\colorbox{red}{\textcolor{white}{SOLUTION MISSING!}}
\section{ -- 3.3}
\section{ -- 3.4 CIDR}
\subsection{3.4.1}
Can the addresses be aggregated?\\
\begin{align*}
64.52.\text{~~}96.0/22 &= 64.52.\text{~~}96.0 - 64.52.\text{~~}99.255 \\
64.52.104.0/22 &= 64.52.104.0 - 64.52.107.255 \\
64.52.112.0/22 &= 64.52.112.0 - 64.52.115.255 \\ 
64.52.120.0/22 &= 64.52.120.0 - 64.52.123.255
\end{align*}
Maximaler Abstand von 24 Class-C-Netzen, also ist die maximale Präfixlänge in
der alle Adressen vorkommen /19.\\
Das Netz 64.52.96.0/19 geht von 64.52.96.0 - 64.52.127.255, aggregiert also alle o.g. Adressen. Allerdings sind darin auch einige andere /22 Netze enthalten, die nicht oben aufgelistet sind.

\subsection{3.4.2}
\begin{align*}
155.46.56.0/22 &\rightarrow \text{Interface 0 (155.46.56.0 - 155.46.59.255)}\\
155.46.60.0/22 &\rightarrow \text{Interface 1 (155.46.60.0 - 155.46.63.255)}\\
180.53.40.0/23 &\rightarrow \text{Router 1 (180.53.40.0 - 180.53.41.255)}\\
\text{default} &\rightarrow \text{Router 2}
\end{align*}
\begin{multicols}{3}
\begin{compactenum}
\item 155.46.63.10 $\rightarrow$ Interface 1
\item 155.46.57.14 $\rightarrow$ Interface 0
\item 155.46.52.2  $\rightarrow$ Router 2
\item 180.53.40.7  $\rightarrow$ Router 1
\item 180.53.56.7  $\rightarrow$ Router 2
\end{compactenum}
\end{multicols}
\section{ -- 3.5 ARP}
\subsection{3.5.1 Wie funktioniert ARP?}
\begin{tabular}{l|l|l}
Beispiel        &    IP      &           MAC \\ \hline
Rechner A &    10.0.0.1     &       AA:11:00:11:00:11 \\
Rechner B &    10.0.0.42      &     BB:42:42:42:42:42\\
\end{tabular}\\
Wenn ein Rechner A in einem LAN ein IP-Paket an einen anderen Rechner B
senden möchte, muss er dazu zunächst die MAC-Adresse des anderen Rechners
herausfinden, da auf Ethernet-Ebene Pakete über die MAC und nicht über die IP
adressiert wird.
Dazu sendet er zunächst ein ARP-Request. Dieser geht auf MAC-Ebene an die
Broadcast-Adresse, wird also von allen Rechnern im gleichen Netzsegment
empfangen. Im ARP-Request steht die IP-Adresse von Rechner B.
\begin{tabular}{c|r l r l}
\multirow{2}{*}{MAC-Layer}  &  from & AA:11:00:11:00:11 \\
           &   to  & ff:ff:ff:ff:ff:ff (broadcast) \\ \hline
\multirow{2}{*}{ARP}   &      fromIP& 10.0.0.1   & fromHW & AA:11:00:11:00:11 \\
      &      to IP &  10.0.0.42 &  toHW &   ff:ff:ff:ff:ff:ff \\
\end{tabular}\\
Wenn Rechner B den Request erhält, erkennt er, dass er betroffen ist, und sendet eine ARP-Response an Rechner A, die wie folgt aussieht:\\
\begin{tabular}{c|r l r l}
\multirow{2}{*}{MAC-Layer} &  from&  BB:42:42:42:42:42 \\
          &  to  &  AA:11:00:11:00:11 \\ \hline
\multirow{2}{*}{ARP}      & fromIP & 10.0.0.42 &  fromHW & BB:42:42:42:42:42 \\
          &  toIP & 10.0.0.1  & toHW  &  AA:11:00:11:00:11\\
\end{tabular}

\subsection{3.5.2 Definition von ARP-Cache-Poisoning}
ARP-Spoofing\,/\,Poisoning bezeichnet das Senden von gefälschten ARP-Paketen, um die ARP-Tabellen so zu verändern, dass der Datenverkehr zwischen zwei Systemen (eines davon oft das Default-Gateway) über ein drittes System umgeleitet wird und somit abgehört und manipuliert werden kann. (Man-in-the-Middle-Angriff).

Eine legale Anwendungsmöglichkeit ist ein schnelles Failover in Hochverfügbarkeitsumgebungen, dabei haben mehrere Rechner die gleiche IP-Adresse und es wird im Fall eines Ausfalls auf MAC-Ebene zum Ersatzsystem umgeleitet.
\\
Quelle: https://de.wikipedia.org/wiki/ARP-Spoofing


\subsection{3.5.3 Unterschied zwischen ARP und RARP}

Mit ARP lässt sich bei bekannter IP die MAC-Adresse herausfinden, mit RARP genau anders herum, also bei bekannter MAC die IP-Adresse.

ARP benötigt hierzu auf niedrigerer Protokollebene (MAC\,/\,Ethernet) einen
Broadcast, da nur die Adresse eines Protokolls auf höherer Ebene (IP) bekannt ist. Bei RARP ist kein Broadcast nötig, denn es ist die MAC-Adresse bekannt und somit lässt sich direkt ein Ethernet-Paket an das Zielsystem senden. Dieses antwortet dann ebenfalls per direktem Paket mit der IP-Adresse.
\section{3.6}
3.6

\subsection{3.6.1}
a) Bei 10.000 Routern muss jeder dieser Router in seiner Routingtabelle 9.999 Einträge haben, insgesamt also 99.990.000
b) 2-Level-Hierarchie: 400 Router pro Subnetz, jeder Router hat 399 Einträge in der Tabelle für das eigene Subnetz und 24 Einträge für die anderen Subnetze, also insgesamt 423 Einträge
c) 3-Level-Hierachie: Da es $20 \cdot 5 = 100$ Regionen gibt, enthält jede Region 100 Router, von denen jeder also 99 Einträge für die benachbarten Router, 4 Einträge für die Subnetze der anderen Regionen und 19 Einträge für die Netze der anderen Cluster enthält. Insgesamt sind dies 122 Einträge pro Router.

\end{document}
